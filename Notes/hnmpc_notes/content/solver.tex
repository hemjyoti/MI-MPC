The QPs arising in both NMPC formulations are addressed using the high-performance interior-point method (\texttt{HPIPM}) solver that is built on top of the linear algebra package \texttt{BLASFEO}. This Riccati-based solver implements an efficient method for the solution of linear-quadratic (LQ) control problem, that is a special instance of equality constrained quadratic program. The corresponding Karush-Kuhn-Tucker (KKT) system is sparse and structured, and this structure is exploited by \texttt{HPIPM} which grants a reduction in the number of flops. As for the linear algebra library, the major difference between existing high-performance implementations of BLAS-- and LAPACK--like routines for embedded applications and \texttt{BLASFEO} is that the last is optimized for small to medium scale matrices. The \texttt{X64\_INTEL\_HASWELL} implementation of \texttt{BLASFEO} package was used, which exploits a set of vectorized instructions for the target CPU. 

Additionally, for the NMPC solutions, we used partial condensing \texttt{HPIPM}-based technique. 
This approach reformulates the large and sparse Hessians of (\textbf{NLP$_1$}) and (\textbf{NLP$_2$}) into small and dense ones, which have a more suitable form for the QP solver. 