In the simulations we assess the effectiveness of the mixed-initiative control algorithm in terms of whether or not human operators are afforded the freedom that is required to modify the reference tracked by the robot, regardless of the level of piloting skill. Additionally, we would like to measure the performance of the navigation task for the two adaptation criteria presented. Figure \ref{fig:l0_a1}--\ref{fig:l2_a2} show the results for six different simulations in which Algorithm 1 was used. Independently of the criterion used to compute $\lambda_{\bullet}$, the values $\lambda_{\bullet} = 0.0, \lambda_{\bullet} \in \mathbb{R} = (0.0,1.0), \lambda_{\bullet} = 1.0$ indicate \emph{fully automatic control}, \emph{shared control} and \emph{fully human control}, respectively.

The graphs depict the evolution of both zone-excursion functions $\lambda_{\bullet}$ for which the insets show the corresponding unfolding of either $\Bar{N}$ or $\epsilon^\star_{max}$. It is also shown the predicted Euclidean distances $\epsilon^\star_i$ analogous to the time interval of the inset. The light blue and pink colors in the graphs for criterion $(A_1)$ indicate the predicted Euclidian distances that are within and exceed the target zone, respectively, e.g. $\Bar{N} = 0$ and $\Bar{N} \in \mathbb{W} = [1,50]$. In contrast, for criterion $(A_2)$ pink corresponds to $\epsilon^\star_{max} \in \mathbb{R} = (0.0,1.0)$.

When $\Bar{N} \in \mathbb{W} = [1,50]$, i.e. PED exceeds the target zone, then $0 <\lambda_{A_1} \leq 1$, resulting in a reduction in the human control authority while $\lim_{\Bar{N}\to 50}\lambda_{A_1}(\Bar{N})$, according to \eqref{eq:weight_matrices}. In general, the insets of $\lambda_{A_1}$ show that, when active, the zone NMPC starts producing a precipitous fall in the human control authority. However, due to the high level of ``trust in the human" embedded in criterion $(A_1)$, this off-peak is short-lived. Because of the settling stage-weighting, the active shooting node is gradually decreased, with respect to previous NMPC solutions, right after the main $\lambda_{A_1}$ off-peak. This decrease corresponds to the transition from the shared control phase, in which $\epsilon^\star_i$ exceeds the target zone, to the phase fully controlled by the human, in which $\epsilon^\star_i$ undulates nonchalantly within $[\check{\epsilon},\Hat{\epsilon}]$, accruing no extra penalization, as $\Bar{N}$ remains equal to zero. 

On the other hand, when using criterion $(A_2)$, the zone NMPC persistently opts for stark shared control throughout the task. The aim of zone-excursion function $\lambda_{A_2}$ is to aggressively reduce human control authority only an increasing in the error margin arises, indicating that this is a criterion with a low level of ``trust in the human". Maximum PED-weighting constantly attenuates human control authority compared with using the fixed nominal weights and, for this reason, is inherently safer as it is ``active" at each prediction step.

The simulation results have shown that the proposed mixed-initiative control algorithm is able to provide to the human operator the freedom to modify the reference tracked by the robot, no matter the level of piloting skill. Proposed zone NMPC with criterion $(A_1)$ yields an oscillatory, albeit slightly more sluggish, return to the target zone. Instead, proposed zone NMPC with criterion $(A_2)$ does not exceed the target zone whatsoever. As maximum PED-weighting ends up being continuously ``active", it is advantageous for more challenging situations, for instance, with an inexperienced pilot, capable of reducing the human workload and also providing higher performance benefits.

%%%%%%%%%%%%%%%%%%%%%%%%%%%%%%%%%%%%%%%%%%%%%%%%%%%%%%%%%%%%%%%%%%%%%%%%%%%%%%%%%%%%%%
%%%%%%%%%%%%%%%%%%%%%%%%%%%%%%%%%%%%%%%%%%%%%%%%%%%%%%%%%%%%%%%%%%%%%%%%%%%%%%%%%%%%%%
\newpage
\paragraph{Inexperienced $(\mathbf{L0})$:} Closed-loop results for mixed-initiative control algorithm.
\begin{figure*}[h]\centering
	\includegraphics[width=1.0\textwidth]{../../Software/plots/case0/noob/position}
	\caption{Using adaptation criterion ($A_1$).}\label{fig:l0_a1}%
	\vspace{0.5cm}
	\includegraphics[width=1.0\textwidth]{../../Software/plots/case1/noob/position}
	\caption{Using adaptation criterion ($A_2$).}\label{fig:l0_a2}%
\end{figure*}
%%%%%%%%%%%%%%%%%%%%%%%%%%%%%%%%%%%%%%%%%%%%%%%%%%%%%%%%%%%%%%%%%%%%%%%%%%%%%%%%%%%%%%
\newpage
\paragraph{Intermediate $(\mathbf{L1})$:} Closed-loop results for mixed-initiative control algorithm.
\begin{figure*}[h]\centering
	\includegraphics[width=1.0\textwidth]{../../Software/plots/case0/inter/position}
	\caption{Using adaptation criterion ($A_1$).}\label{fig:l1_a1}%
	\vspace{0.5cm}
	\includegraphics[width=1.0\textwidth]{../../Software/plots/case1/inter/position}
	\caption{Using adaptation criterion ($A_2$).}\label{fig:l1_a2}%
\end{figure*}
%%%%%%%%%%%%%%%%%%%%%%%%%%%%%%%%%%%%%%%%%%%%%%%%%%%%%%%%%%%%%%%%%%%%%%%%%%%%%%%%%%%%%%
\newpage
\paragraph{Experienced $(\mathbf{L2})$:} Closed-loop results for mixed-initiative control algorithm.
\begin{figure*}[h]\centering
	\includegraphics[width=1.0\textwidth]{../../Software/plots/case0/pro/position}
	\caption{Using adaptation criterion ($A_1$).}\label{fig:l2_a1}%
	\vspace{0.5cm}
	\includegraphics[width=1.0\textwidth]{../../Software/plots/case1/pro/position}
	\caption{Using adaptation criterion ($A_2$).}\label{fig:l2_a2}%
\end{figure*}
%%%%%%%%%%%%%%%%%%%%%%%%%%%%%%%%%%%%%%%%%%%%%%%%%%%%%%%%%%%%%%%%%%%%%%%%%%%%%%%%%%%%%%
%%%%%%%%%%%%%%%%%%%%%%%%%%%%%%%%%%%%%%%%%%%%%%%%%%%%%%%%%%%%%%%%%%%%%%%%%%%%%%%%%%%%%%