The fundamental problem in mixed-initiative control is how to blend human inputs and automatic controller commands to realize the former as much as possible while always enforcing safety. In this context, the formulation of our problem hinges upon the following components:
\begin{itemize}
	\item A \emph{robot} represented by a particular nonlinear, time-varying system
\begin{equation}
	\dot{\xi} = f(\xi,u),\label{eq:robot_dynamics}
\end{equation}
subject to a set of constraints, where states, control inputs and dynamics are denoted by $\xi \in \mathbb{R}^{n_\xi}$, $u \in \mathbb{R}^{n_u}$ and $f \vcentcolon \mathbb{R}^{n_\xi} \times \mathbb{R}^{n_u} \rightarrow \mathbb{R}^{n_\xi}$, respectively. 
	\item Task variables denoted by $\eta = z(\xi)$ and described in $\mathbb{R}^{n_\eta}$. Typical examples are:
	\begin{itemize}
	\item a subset of the state variables (which may even coincide with the state itself), e.g., cartesian points.
	\item some nonlinear vector, e.g., kinematic map for some point of the robot.
	\end{itemize}
	\item A motion generator that provides the collision-free \emph{reference trajectory} $\eta^r \in \mathbb{R}^{n_\eta}$ for the task variables. It could be either a global planner or an autonomous controller.                                                                                                                                                                                                                                                                                                                                                                                                                                                                                                                                                                                                                                                                                                                                                                                                                                                                                                                                                                                                                                                                                                                                                                                                                                                                                                                                                                                                                                                                                                                                                                                                                                                                                                                                                                                                                                                                                                                                                                                                                                                                                                                                                                                                                                                                                                                                                                                                                                                                                                                                                                                                                                                         
	\item The human who specifies desired values to the control interface variables denoted by $\nu^r \in \mathbb{R}^{n_\nu}$. These values are henceforth called \emph{human inputs}. As the notation suggests, they are reference signals representing human intentions.
	\item A set of \emph{working conditions} defining the minimum requirements that the robot must meet to operate healthily. They can contain both safety rules (e.g., maximum kinetic energy, maximum dissipation, etc.) and/or task-related rules (e.g., the accuracy of end-effector positioning and orientation, etc.).
	\begin{assumption}
		Working conditions are defined such that the resulting set is convex.
	\end{assumption}
\end{itemize}
For this setting, we want to devise a \emph{mixed-initiative controller} that will attempt to execute human inputs as much as possible without compromising the working conditions and the set of constraints inherent to the robot's physical limitations. As we will see, in addition to the mixed-initiative controller itself, the control algorithm also includes a blending mechanism that predicts the violation of working conditions and lends most of the control authority to the most capable agent at any time.

Let us now delve into the case of safe human-quadrotor interaction. The human operates the quadrotor providing it with four inputs: the desired roll, pitch, $z$ angular rate, and total thrust. They are sent to the robot through a control interface, which is usually a joystick. Here, the working conditions concern safety that comes in the form of collision-free motions. Thus, for the human pilot to safely fly a quadrotor, he/she needs to perceive the surrounding cues and decide how to control the quadrotor by modifying the inputs above to avoid potential collisions. This job is always demanding for a novice, and in most cases, the quadrotor ends up crashing in the workspace \cite{xu2018}. That is why a mixed-initiative controller capable of guaranteeing safety during the learning process is of particular importance. Another key aspect of this process is repetition. Through repetition, the pilot can gradually reach a motor pattern that enables him/her to execute the task and increase his/her competence successfully. To that end, we assume that the human-quadrotor system must track a position reference repeatedly. This scenario will help us to illustrate the performance of our algorithm.   
  
%  A common physical limitation in quadrotor systems is that propellers can only generate bounded thrust. More sophisticated schemes include control at the rotor torque level yielding a high-dimensional state space that can be challenging to treat.