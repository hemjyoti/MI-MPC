Quadrotors have been widely used across various applications, resulting in a significant market expected to grow exponentially. Due to this increasing demand, one can envisage that, in the future, learning to fly a quadrotor will become a taken-for-granted skill, just as learning to drive a car nowadays. A first step could be a mixed-initiative control approach that guarantees the aerial system's ``working conditions" while the novice operator learns to fly it. In other words, as long as some safety- and/or task-related rules are met, the operator commands are obeyed. Once novices start working in the context of complex tasks, it is essential to provide them with the assistance that gradually decreases as their competence increases -- here, competence refers to the knowledge and ability that allow the use of the robot to accomplish a task. Conversely, one could think of an experienced operator who already has the competence level that enables him/her to fly a quadrotor. But as his/her cognitive load is primarily focused on the short-term aspect of the task, he/she cannot consider other underlying factors in the long run (e.g., safety). It is then essential to have a control approach that supervises the task in the long term to prevent the operator from being overwhelmed by (too) high engagement.  

To fill these gaps, in this paper, we propose an efficient mixed-initiative controller based on nonlinear model predictive control (NMPC) to enforce safety in human-quadrotor interactions. The technical difficulties associated with the proposed controller are twofold: first, satisfy both the existing constraints and the human intentions; second, solve the underlying optimization problems that include a high-dimensional quadrotor system under the available computation time. We demonstrate how to handle these difficulties using concepts of zone MPC and online weight adaptation within the real-time iteration (RTI) scheme, an efficient Newton-type algorithm for real-time NMPC applications. To further speed-up solution times, we exploit high-performance numerical optimization algorithms, structure-exploiting convex solver, and linear algebra library in our implementation. The mixed-initiative controller is validated in simulation, where an autonomous algorithm emulates the behavior of pilots with different competence levels. To the best of our knowledge, this is the first mixed-initiative controller that overcomes all aforenamed dilemmas. For this reason, this work makes a significant contribution to human-robot interaction literature.
