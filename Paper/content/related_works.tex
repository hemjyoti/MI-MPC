One way robots and humans can interact is to put the human in the loop with some blending scheme \cite{amirshirzad2019}. In a nutshell, the human and the automatic controller have control over the robot, and both systems must adapt to ensure task completion. For example, in bimanual robot manipulation, the robot can arbitrate between the user's command inputs and its underlying motion policies and understanding of bimanual tasks \cite{rakita2019, xi2019}. In exoskeleton systems, the interplay between robot and human may be characterized as an interaction between teacher and student in a learning process. The teacher (robot) tries to minimize the student (human) error, applying a minimal effort \cite{bao2020, santos2019}. In the context of autonomous cars, the driver interacts with the automatic controller that aims at reducing the driver's workload and, at the same time, taking prompt actions in case of human failure \cite{huang2019, lu2019}. For human-swarm systems, the interaction paradigm is less obvious. As the size of the swarm increases, control should become more focused on the swarm as a whole rather than on the individuals, given the human's limited capacity to multitask. In this case, the interaction is more concerned with the human-swarm ability to accomplish a particular task than with the spatial positioning of the swarm \cite{ashcraft2019, nam2019}. 

Other interactive approaches rely on methods designed from the human perspective, where, in general, the presence of haptic cues increases situational awareness. These methods lean heavily in favor of perceiving the human as the source of action and the robot as the passive collaborator. For instance, virtual fixtures have been used to inform the operator of the highest comfort position, distance from the target position, or proximity to unsafe kinematic configurations \cite{rahal2020, abi2020}. Obstacle avoidance is one of the most critical requirements to be met in many robotics scenarios. In this matter, haptic feedback can help inform the operator of instantaneous collisions \cite{parsa2020, masone2018}. The work in \cite{franchi2017} compares several human-collaborative schemes, highlighting their main aspects where haptic feedback is one of those.
